\chapter{まえがき}
\label{chap:chap00-preface}
\begin{reviewimage}%%gmail
\includegraphics[width=\maxwidth]{./images/chap00-preface/gmail.png}%
\reviewimagecaption{Gmailに出てくる?マーク}
\label{image:chap00-preface:gmail}
\end{reviewimage}

Gmailにて、このようなメールが届いたことありますか?\\
または、試しに会社や取得したドメインからメールをGmailアドレスメール送信してみてください。\\
Gmailで表示される?マークはスパムメールとして判定される前段階の状態、このまま放置しておくとスパムメールとしてメールが届かなくなります。この状態にならないため早めの対策が必要です。\\
メールの仕組みと歴史的な経緯もありメールを送信した場所(ドメイン)とメールサーバーの正当性を担保することはメールサーバーだけでは不可能です。これらの技術はSPF\footnote{SPF(Sender Policy Framework)とは、送信ドメイン詐称を防いで正当なドメインから送信を検証する仕組み}とDKIM\footnote{DKIM(Domainkeys Identified Mail)とは、送信ドメインを電子署名で認証する仕組み}によって行われます。

本書は、メールの基本的な構造とメール送信で必要不可欠なSPFやDKIMの話を、わかりやすく解説した本です。この本を読めばメールのことを深く知り、そしてSPFやDKIMの基礎的なことが身につき送信するメールがスパムメールにされなくなります。

今後も使われるメールを長く付き合うためにも本書を読んでいただけたら幸いです。

\paragraph*{本書で得られること}
\label{sec:-0-0-0-1}

\begin{itemize}
\item メールの基本的な構造の知識
\item メール送信の正当性についての知識
\item SPFについての基礎的な知識
\item DKIMについての基礎的な知識
\item スパムメールにならないための設定知識
\end{itemize}

\paragraph*{対象読者}
\label{sec:-0-0-0-2}

\begin{itemize}
\item ドメインを取得したけど、メールを使っていない人
\item 会社のメールアドレスからメールを出しても、クライアントから届かないと言われて困っている人
\item フリーランスで仕事をはじめたけど、メールが届くか心配な人
\end{itemize}

\paragraph*{前提知識}
\label{sec:-0-0-0-3}

\begin{itemize}
\item メール配信の基本的な知識
\item ドメイン取得する知識
\item ドメイン設定を変更できるだけの知識
\item DNSに関する簡単な知識
\end{itemize}

\paragraph*{問い合わせ先}
\label{sec:-0-0-0-4}

\begin{itemize}
\item URL: https://sapi{-}kawahara.netlify.com/
\item Twitter: @sapi\textunderscore{}kawahara
\end{itemize}
